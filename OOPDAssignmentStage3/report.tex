\documentclass{article}

\begin{document}
\begin{center}
    \Huge
    OOPD Assignment Stage 3 Report\\
    \large
    Jakob Wyatt\\
    19477143
\end{center}

\section{Simplification of Design using Inheritance}
The first way that inheritance simplified the design was
in the Record, Digital, and Cassette classes. Many of the 
methods that were duplicated between these classes could be moved
into the Music superclass instead, and some code within the constructors
could be delegated to the super constructor instead.\\
The second way that inheritance simplified the design was in the 
MusicCollection class; specifically, the use of arrays to store the
music objects. By using a single array to store the music objects,
instead of three, much of the codebase in MusicCollection could be
simplified. The first example of this is in the getters and setters.
As only one array now existed, MusicCollection only required one 
getter and setter, instead of the three equivalent ones it used previously.
The search and display algorithms were also simplified, as they only required
one loop instead of three.

\section{Complications introduced by Inheritance}
There was only one complication introduced by implementing inheritance.
This was the implementation of a CSV constructor for Music sub-classes
such as Record, Digital and Cassette.\\
The reason why this was so difficult
to implement was because java requires the super constructor to be the first
statement in the method. However, checking must be done on the csv string
before some of it is passed to the super constructor
(e.g. Null checks, valid array index checks).\\
However, due to the mutability
of Music (as required in the assignment brief), this problem was rather
simple to solve. Instead of having a CSV constructor in Music, there was
instead a method that input CSV data and set class fields to this data.
This meant that the sub class CSV constructors could call the default
super constructor, make any checks required, and then call the input CSV
method on the super class.

\section{Justification of Down Casting}
The only down casting present in this code are the equals
methods, when down casting from the Object type to the class
type.
This is unavoidable, as getter methods must be called
on the Object to determine if it is equal. The downcasting also
only occurs after it has been determined that the Object
is an instance of the class. This is also the canonical way
of implementing the equals method.

\end{document}

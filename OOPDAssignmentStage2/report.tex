\documentclass{article}

\begin{document}
\begin{center}
    \Huge
    OOPD Assignment Stage 2 Report\\
    \large
    Jakob Wyatt\\
    19477143
\end{center}

For the purposes of this document, a "client" is defined as the programmer
using a class. A "user" is defined as the person using the final program.

\section{Justification of Container Class Functionality}
There were three container classes used in this project:
Digital, Cassette, and Record. There was also one controller class,
MusicCollection, which managed arrays of these three classes.\\
All functionality, where possible, was added to the container classes.
However, there were three notable exceptions to this.\\
The first is when user IO was required. All user IO
was done within the UserInterface
class, with the container classes returning strings rather than printing
the values themselves. The advantage of this approach is that it is more
flexible, as the client can determine if the values are output to the user,
logged, used for debugging, or output to a file. Another advantage of this
approach is that it helps guarentee that class methods cause no side-effects
outside of the class. This is useful to the client, as it makes it easier
to code when side effects do not need to be considered.\\
The second is when file IO was required. All file IO was done within the
FileManager class, which was called by UserInterface to read from and 
write to files. Any file data was passed in and out of the container classes
via string arrays. This was done for the same reasons as mentioned above, 
as it improves flexibility and decreases side-effects within a class.\\
The third is when multiple instances of the same class must be used.
An example of this is duration (within MusicCollection),
which uses arrays of Music objects to find the total duration of the library.
This functionality is not implemented within the container classes,
as it requires the container classes to assume how the client stores their
data. Although arrays are the only storage mechanism in this progam,
container storage classes such as ArrayList are frequently used
in other java code. A way around this is to use the Iterable interface;
however, this is outside of the scope of this unit, and incompatible with
arrays.

\section{Issues with Object Arrays}
The main issue that was encountered while using object arrays was
ensuring that object arrays could not be modified without the knowledge of
the class. This was done by cloning every object either entering or exiting
the object, and using or returning the cloned object instead. This ensures
that the object cannot be modified, however, it complicates program structure
and makes some code slower, as cloning is not as fast as passing an object
reference. A way around this is to make the container classes immutable;
however, this is not allowed in this unit.

\section{Issues with Managing three Arrays}
The main issue with managing three seperate arrays was the massive amount of
code duplication required. Instead of looping through all Music objects at
once, three loops had to be implemented, with one per array. This increased
the amount of code, and therefore the potential for bugs. It also made
the code more difficult to read,
as it was difficult to identify the important
sections of a method. It would be simpler to use a single array, as
looping through all Music objects would only require one for loop. It would
also reduce the number of member functions in the controller class,
as a superclass type could be returned rather than three seperate methods
doing the same thing. This could be achieved via inheritance, which is used
in Stage 3 of the assignment.

\end{document}